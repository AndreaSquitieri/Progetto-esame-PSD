%%%%%%%%%%%%%%%%%%%%%%%%%%%%%%%%%%%%%%%%%
% Wenneker Assignment
% LaTeX Template
% Version 2.0 (12/1/2019)
%
% This template originates from:
% http://www.LaTeXTemplates.com
%
% Authors:
% Vel (vel@LaTeXTemplates.com)
% Frits Wenneker
%
% License:
% CC BY-NC-SA 3.0 (http://creativecommons.org/licenses/by-nc-sa/3.0/)
% 
%%%%%%%%%%%%%%%%%%%%%%%%%%%%%%%%%%%%%%%%%

%----------------------------------------------------------------------------------------
%	PACKAGES AND OTHER DOCUMENT CONFIGURATIONS
%----------------------------------------------------------------------------------------

\documentclass[11pt]{scrartcl} % Font size

\input{structure.tex} % Include the file specifying the document structure and custom commands

%----------------------------------------------------------------------------------------
%	TITLE SECTION
%----------------------------------------------------------------------------------------

\title{	
	\normalfont\normalsize
	\textsc{Università degli studi di Salerno}\\ % Your university, school and/or department name(s)
	\vspace{25pt} % Whitespace
	\rule{\linewidth}{0.5pt}\\ % Thin top horizontal rule
	\vspace{20pt} % Whitespace
	{\huge Progetto PSD}\\ % The assignment title
	\vspace{12pt} % Whitespace
	\rule{\linewidth}{2pt}\\ % Thick bottom horizontal rule
	\vspace{12pt} % Whitespace
}

\author{
	\LARGE
	Robustelli Renato \\
	\LARGE
	Squitieri Andrea \\
	\LARGE
	Vitolo Angelo
} % Your name

% \date{\normalsize\today} % Today's date (\today) or a custom date
\date{\null}

\begin{document}

\maketitle % Print the title

%----------------------------------------------------------------------------------------
%	FIGURE EXAMPLE
%----------------------------------------------------------------------------------------

\section{Introduzione}


%------------------------------------------------


Il progetto si concentra sulla creazione di un sistema di gestione delle conferenze, con l'obiettivo di organizzare e gestire eventi e stanze in modo efficiente. Il sistema è progettato per facilitare la pianificazione, la modifica, la visualizzazione e l'archiviazione di eventi e stanze all'interno di una conferenza.

\subsection{Obiettivi del progetto}
Gli obiettivi principali del progetto includono:

\begin{itemize}
				\item  Gestione di Conferenze: Possibilità di aggiungere, modificare e rimuovere eventi e stanze di una conferenza.
				\item Assegnazione delle Stanze: Funzionalità per assegnare stanze a specifici eventi.
				\item Visualizzazione delle Informazioni: Strumenti per visualizzare gli eventi ordinati per data d'inizio.
				\item Persistenza dei Dati: Capacità di salvare e caricare i dati delle conferenze da file per garantire la persistenza delle informazioni.
\end{itemize}

%----------------------------------------------------------------------------------------
%	TEXT EXAMPLE
%----------------------------------------------------------------------------------------

\section{Motivazione della scelta degli ADT}

\subsection{Motivazioni per l'uso dell'ADT Date}


Nel progetto di gestione delle conferenze, una delle componenti fondamentali è la gestione delle date. Gli eventi della conferenza devono essere pianificati in specifici giorni e orari, e per garantire una gestione precisa e coerente di queste informazioni, è stato sviluppato e utilizzato un ADT chiamato Date. Utilizzando questo ADT possiamo garantire:
\begin{itemize}
				\item  \textbf{La validità delle date}:  Le date inserite nel sistema saranno sempre valide (es. nessuna data con un mese o giorno non esistente).
				\item \textbf{Il corretto confronto fra date}: Il confronto fra due date avvenga in maniera corretta e semplice.
\end{itemize}
Nello specifico, l'ADT Date viene usato all'interno del progetto per rappresentare le date d'inizio e fine di ogni evento, offrendo una precisione al minuto.

\subsection{Motivazione dell'uso dell'ADT Room}


L'ADT Room è stato introdotto per gestire le informazioni relative alle sale all'interno del sistema di gestione delle conferenze. Esso è stato progettato per rappresentare una sala conferenze con attributi chiave quali il nome della sala, il numero di posti e l'eventuale disponibilità.



\subsection{Motivazione dell'uso dell'ADT RoomList}

L'ADT RoomList è stato introdotto per gestire in modo efficiente un insieme di sale conferenze all'interno del sistema di gestione delle conferenze. Esso è stato progettato per rappresentare una collezione di oggetti Room, permettendo di mantenere organizzate e facilmente accessibili tutte le informazioni relative alle diverse sale conferenze.

\subsection{Motivazione dell'uso dell'ADT Event}

L'ADT Event è stato introdotto per gestire in modo efficiente e organizzato le informazioni relative agli eventi all'interno del sistema di gestione delle conferenze. Esso è stato progettato per rappresentare un singolo evento, includendo dettagli cruciali come il nome dell'evento, la data e l'ora di inizio e fine, e la sala assegnata.

\subsection{Motivazione dell'uso dell'ADT EventBst}

L'ADT EventBst è stato introdotto per gestire in modo efficiente e ordinato la raccolta di eventi all'interno del sistema di gestione delle conferenze.Esso rappresenta un albero binario di ricerca in cui ogni nodo contiene un singolo evento e le relazioni di ordinamento sono definite sulla base delle proprietà degli eventi, come la data e l'ora di inizio e, nel caso in cui queste vengano condivise da altri eventi, il nome.\\
L'uso di un albero binario di ricerca permette di aggiungere eventi con una complessità temporale media di $O(\log{n})$ ma, per favorire l'utilizzabilità del software, le operazioni di modifica e rimozione non possono sfruttare le proprietà dell'albero e sono quindi di complessita $O(n)$ (Esse verranno sempre precedute da una ricerca lineare per ID dell'evento).\\
Ciò nonostante, riteniamo che l'uso di questo ADT sia giustificato, dato che le possibili alternative avrebbero comunque mantenuto una complessità temporale di $O(n)$ nella modifica e nella rimozione ma avrebbero reso lineare anche la complessità dell'aggiunta degli eventi.

\subsection{Motivazione dell'uso dell'ADT Conference}

L'ADT Conference è stato implementato per gestire in modo completo e strutturato l'organizzazione di una conferenza, inclusi gli eventi, le sale conferenze e le loro assegnazioni. Esso è progettato come un'astrazione completa della conferenza, che comprende una collezione di eventi, una lista delle sale conferenze disponibili e le assegnazioni degli eventi alle sale.\\
Grazie a questa astrazione, ci assicuriamo che ciascun evento e sala siano identificati da ID univoci e che non vi siano sovrapposizioni temporali tra gli eventi assegnati alla stessa sala.


\section{Progettazione}
\section{Specifiche sintattiche e semantiche}
\section{Razionale dei casi di test}
%------------------------------------------------


\end{document}
